% !TEX encoding = UTF-8 Unicode
%!TEX root = thesis.tex
% !TEX spellcheck = en-US
%%=========================================
\chapter{Conclusions and Future Work}
In our project we devised a visualization method for large ontology
matching that has the following features:
\begin{enumerate}
	\item Representation Dimension
	\begin{itemize}
		\item Visual representation for source and
                  target ontologies and for mappings between classes
                  in those ontologies. 
		\item Focus on a small part of the ontologies at a
                  time allowing to quickly access other parts of the ontologies.
                \item Display of the similarity values between classes
                  and provide an overview of the similarity values for the
                  children of those classes.
	\end{itemize}
	\item Interaction Dimension
	\begin{itemize}
	 	\item User-driven navigation of classes and properties.
		\item Ability to search for a specific class and to
                  traverse the ontologies vertically (parents and
                  children of a class) and horizontally (siblings of a
                  class) .
	\end{itemize}
	\item Analysis and Decision Making
	\begin{itemize}
		\item Display several possible mappings to the users, so that
                  they can choose among them, as part of a user
                  feedback loop strategy that combines automatic with
                  manual matching methods. 
		\item Manages the mapping decisions made by the users,
                  by keeping track of them.
	\end{itemize}
        \item Implementation
	\begin{itemize}
        \item Uses JavaFX.
       \item Integrates with AgreementMaker, therefore can be used in
         conjunction with other visualization methods.
       \end{itemize}
\end{enumerate}

Clearly, there are several directions for future work. The first one
is that we would like to extend the comparison of matching algorithms
to more than two at a time. Figure~\ref{fig:updated_structure} shows our preliminary design to
compare more than two algorithms. Each node in the structure is a
pie chart that can be traversed recursively. The center pie chart
represent the current class and the outer pie charts are the matching
results of different matching algorithms. We connect the center and
outer charts with a line: the longer the line, the smaller the matching confidence value. 

\begin{figure*}[!ht]
	\centering
	\includegraphics[width=4.5in]{pics/structure.png}
	\caption{Comparison of several matching algorithms.}
	\label{fig:updated_structure}
\end{figure*}

It may be the case that no single visualization strategy works
separately, especially for very large ontologies. AgreementMaker already provides several different
strategies~\cite{cruz-icde-demo,vldb2009demo}. Experiments would be needed to determine the usability and
effectiveness of
the different strategies when used separately or in coordination with
one another.



